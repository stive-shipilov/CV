
\documentclass[letterpaper,11pt]{article}

\usepackage{latexsym}
\usepackage[empty]{fullpage}
\usepackage{titlesec}
\usepackage{marvosym}
\usepackage[usenames,dvipsnames]{color}
\usepackage{verbatim}
\usepackage{enumitem}
\usepackage[hidelinks]{hyperref}
\usepackage{fancyhdr}
\usepackage[english, russian]{babel}
\usepackage{tabularx}
\usepackage{indentfirst}
\setlength{\parindent}{1cm}

\input{glyphtounicode}


%----------FONT OPTIONS----------
% sans-serif
% \usepackage[sfdefault]{FiraSans}
% \usepackage[sfdefault]{roboto}
% \usepackage[sfdefault]{noto-sans}
% \usepackage[default]{sourcesanspro}

% serif
% \usepackage{CormorantGaramond}
% \usepackage{charter}


\pagestyle{fancy}
\fancyhf{} % clear all header and footer fields
\fancyfoot{}
\renewcommand{\headrulewidth}{0pt}
\renewcommand{\footrulewidth}{0pt}

% Adjust margins
\addtolength{\oddsidemargin}{-0.5in}
\addtolength{\evensidemargin}{-0.5in}
\addtolength{\textwidth}{1in}
\addtolength{\topmargin}{-.5in}
\addtolength{\textheight}{1.0in}

\urlstyle{same}

\raggedbottom
\raggedright
\setlength{\tabcolsep}{0in}

% Sections formatting
\titleformat{\section}{
  \vspace{-4pt}\scshape\raggedright\large
}{}{0em}{}[\color{black}\titlerule \vspace{-5pt}]

% Ensure that generate pdf is machine readable/ATS parsable
\pdfgentounicode=1

%-------------------------
% Custom commands
\newcommand{\resumeItem}[1]{
  \item\small{
    {#1 \vspace{-2pt}}
  }
}

\newcommand{\resumeSubheading}[4]{
  \vspace{-2pt}\item
    \begin{tabular*}{0.97\textwidth}[t]{l@{\extracolsep{\fill}}r}
      \textbf{#1} & #2 \\
      \textit{\small#3} & \textit{\small #4} \\
    \end{tabular*}\vspace{-7pt}
}

\newcommand{\resumeSubSubheading}[2]{
    \item
    \begin{tabular*}{0.97\textwidth}{l@{\extracolsep{\fill}}r}
      \textit{\small#1} & \textit{\small #2} \\
    \end{tabular*}\vspace{-7pt}
}

\newcommand{\resumeProjectHeading}[2]{
    \item
    \begin{tabular*}{0.97\textwidth}{l@{\extracolsep{\fill}}r}
      \small#1 & #2 \\
    \end{tabular*}\vspace{-7pt}
}

\newcommand{\resumeSubItem}[1]{\resumeItem{#1}\vspace{-4pt}}

\renewcommand\labelitemii{$\vcenter{\hbox{\tiny$\bullet$}}$}

\newcommand{\resumeSubHeadingListStart}{\begin{itemize}[leftmargin=0.15in, label={}]}
\newcommand{\resumeSubHeadingListEnd}{\end{itemize}}
\newcommand{\resumeItemListStart}{\begin{itemize}}
\newcommand{\resumeItemListEnd}{\end{itemize}\vspace{-5pt}}

%-------------------------------------------
%%%%%%  RESUME STARTS HERE  %%%%%%%%%%%%%%%%%%%%%%%%%%%%


\begin{document}

%----------HEADING----------
% \begin{tabular*}{\textwidth}{l@{\extracolsep{\fill}}r}
%   \textbf{\href{http://sourabhbajaj.com/}{\Large Sourabh Bajaj}} & Email : \href{mailto:sourabh@sourabhbajaj.com}{sourabh@sourabhbajaj.com}\\
%   \href{http://sourabhbajaj.com/}{http://www.sourabhbajaj.com} & Mobile : +1-123-456-7890 \\
% \end{tabular*}

\begin{center}
    \textbf{\Huge \scshape Шипилов Степан Юрьевич} \\ \vspace{1pt}
    \small +7 925 037 31 45 $|$ \href{mailto:stive.shipilov@yandex}{\underline{stive.shipilov@yandex}} $|$ 
    \href{https://t.me/stivek11}{\underline{stivek11}} $|$
    \href{https://github.com/stive-shipilov}{\underline{github.com/stive-shipilov}}
\end{center}


%-----------EDUCATION-----------
\section{Образование}
\resumeSubHeadingListStart
  \resumeSubheading
    {МФТИ - Московский физико-технический университет}{Москва}
    {Физтех-школа радиотехники и комьютерных технологий}{2023 -- 2027 $|$ 2 курс $|$ $\textbf{GPA: 9.1/10}$}
\resumeSubHeadingListEnd

\vspace*{-3mm} % Уменьшение вертикального отступа между блоками
\resumeSubHeadingListStart
  \resumeSubheading
    {Основные курсы:}{}
    {\parbox{0.8\textwidth}{Математический Анализ, Линейная Алгебра, Информатика, Физика, Дифференциальные уравнения, Теория Вероятности, Математическая статистика, Базы данных и SQL, Машинное обучение и анализ данных. Курс от Сбера:  ML и управление рисками}}{}
\resumeSubHeadingListEnd


%-----------PROJECTS-----------
\section{Проекты}
    \resumeSubHeadingListStart
      \resumeProjectHeading
          {\textbf{Библиотека для обработки данных физического эксперимента} {\color{blue}\href{https://github.com/stive-shipilov/ML-and-statistic-for-phisic-experiment}{[GitHub]}} $|$ \emph{Python, Pandas, Numpy, SciPy}}
          
          \resumeItemListStart
            \resumeItem{Была разработана библиотека на Python для автоматизированной обработки данных экспериментов.}
            \resumeItem{Реализованы продвинутые методы регрессионного анализа для линейных и сложных функций.}
            \resumeItem{Была добавлена визуализация результатов для наглядного представления данных.}
          \resumeItemListEnd
    \resumeProjectHeading
          {\textbf{Исследование методов линейных регрессий} 
          {\color{blue}\href{https://github.com/stive-shipilov/Statistical-methods-research}{[GitHub]}} $|$ \emph{Latex, Python, MatPlotLib, Git}}
          
          \resumeItemListStart
            \resumeItem{Было проведено исследование различных методов линейных регрессий с оценкой их эффективности в разных сценариях.}
            \resumeItem{Написал доклад, в котором рассмотрел применения различных методов регрессионного анализа, их плюсы и минусы.}
          \resumeItemListEnd
    \resumeProjectHeading
          {\textbf{База данных института} 
          {\color{blue}\href{https://github.com/stive-shipilov/-University-database-design}{[GitHub]}} $|$ \emph{SQL, Git, Проектирование баз данных}}
          
          \resumeItemListStart
            \resumeItem{Спроектировал схему базы данных для хранения ключевых данных о деятельности учебного заведения, его работников и студентов}
            \resumeItem{База данных находится в 3НФ и поддерживает дальнешую масштабируемость}            
            \resumeItem{Написал тестовые SQL запросы для демонстрации работоспособности базы данных}
            
          \resumeItemListEnd
    \resumeProjectHeading
          {\textbf{Автоматизированная обработка физического эксперимента. VBA} {\color{blue}\href{https://github.com/stive-shipilov/VBA-experiment-processing}{[GitHub]}}  $|$ \emph{VBA, Excel, Git}}
          
          \resumeItemListStart
            \resumeItem{Реализовал приложение на языке макросов Excel, позволяющее обрабатывать большие массивы однотипных данных}
            \resumeItem{Реализовал удобный и быстрый подсчёт погрешности величины, представляющей собой суперпозицию разных величин}
            \resumeItem{Создал удобный интерфейс для взаимодействия с приложением}
          \resumeItemListEnd
    \resumeProjectHeading
          {\textbf{Аналитика данных на SQL} {\color{blue}\href{https://github.com/stive-shipilov/SQL-projects}{[GitHub]}}  $|$ \emph{SQL, Git}}
          
          \resumeItemListStart
            \resumeItem{Включает несколько модулей: работу с матрицами, анализ парковочных данных и исследование покупательской активности, анализ графов.}
            \resumeItem{Реализованы сложные SQL-запросы для обработки, анализа и оптимизации данных, включая математические вычисления, управление парковками и аналитику покупок.}
            \resumeItem{Проект демонстрирует применение SQL в реальных сценариях: от математических операций до бизнес-аналитики.}
          \resumeItemListEnd
    % \resumeProjectHeading
    %       {\textbf{Автоматическое создание расписания на месяц. VBA} {\color{blue}\href{https://github.com/stive-shipilov/Automated-calendar}{[GitHub]}} $|$ \emph{VBA, Excel, Git}}
          
    %       \resumeItemListStart
    %         \resumeItem{Было реализовано автоматизированное создание расписания на VBA, позволяющее формировать расписание на месяц на основе введённых дат и времени работы..}
    %         \resumeItem{Была обеспечена гибкость и автоматизация процесса планирования с использованием макросов Excel (.xlsm), что исключает рутинные операции.}
    %       \resumeItemListEnd
    \resumeProjectHeading
          {\textbf{Автоматическое создание графиков на Python} {\color{blue}\href{https://github.com/stive-shipilov/Graphics-creating}{[GitHub]}}  $|$ \emph{Python, Pandas, Numpy, SciPy, Git}}
          
          \resumeItemListStart
            \resumeItem{Было реализовано построение графиков на основе CSV-данных с использованием Matplotlib.}
            \resumeItem{Была разработана обработка данных с фильтрацией, отображением погрешностей и аппроксимацией.}
            \resumeItem{Была обеспечена удобная визуализация с легендами, подписями осей и автоматическим выводом таблиц.}
          \resumeItemListEnd
    \resumeSubHeadingListEnd
    

%
%-----------О СЕБЕ-----------
\section{О себе}
\resumeItemListStart
  \resumeItem{В школе серьезно занимался силовым видом спорта, выиграл чемпионат Европы}
  \resumeItem{Закончил школу с Российской и Московской золотыми медалями}
  \resumeItem{Вхожу в ТОП 3 рейтинга по академическим успехам по курсу в институте. Получатель Абрамовской стипендии.}
  \resumeItem{Помимо академической программы, прохожу дополнительные курсы по математической статистике, SQL, машинному обучению и анализу данных. Также прохожу курс от Сбера:  ML и управление рисками}
  \resumeItem{Умею работать в условиях строгих сроков и дедлайнов}
  \resumeItemListEnd

% \hspace{1cm}Учась в школе, я участвовал в инженерных олимпиадах, стал призёром в олимпиаде "Инженеры будущего" и "Курчатовcкий проект" :

% \hspace{1cm}В ходе обучения на физтехе я развил навыки работы в условиях строгих сроков и дедлайнов, научился эффективно планировать свою работу и достигать поставленных целей в установленное время. Я привык решать сложные задачи, адаптируясь под новые требования и подходы, что позволяет мне успешно справляться с вызовами. Также активно участвовал в мозговых штурмах, где учился работать в команде и искать нестандартные решения для различных проблем."



%-------------------------------------------
\end{document}